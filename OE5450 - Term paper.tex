\documentclass{article}
\usepackage{amsmath} % For advanced mathematical typesetting
\usepackage{graphicx} % For including graphics
\usepackage{amssymb} % For mathematical symbols
\usepackage[table]{xcolor}
\usepackage{tcolorbox}
\usepackage{float}
\usepackage{array}
\usepackage{verbatim}
\usepackage{longtable}
\usepackage[utf8]{inputenc}
\usepackage{xcolor}
\usepackage{fvextra} % Provides extended Verbatim features
\usepackage[a4paper, margin=1in]{geometry}


% Define a custom Verbatim environment with colored text
\DefineVerbatimEnvironment{ColoredVerbatim}{Verbatim}
{formatcom=\color{red}, breaklines=true} % Replace 'red' with the desired color



\title{\textbf{\Large{Numerical Modeling of Unsteady Open Channel Flows}}}
\author{\textit{Aatmaj Bhayani - NA22B018}} % Replace "Your Name" with your name
\date{\today} % You can replace \today with a specific date, e.g., "November 2024"


\begin{document}

\begin{titlepage}
    \centering
    \vspace*{2cm} % Adjust vertical space to position the title
    {\Huge \textbf{Numerical Modeling of Unsteady Open Channel Flows} \par}
    \vspace{1cm}
    {\Large \textit{by} \par}
    \vspace{0.5cm}
    {\Large Aatmaj Bhayani - NA22B018 \par}
    \vfill
    {\large Term Paper OE5450 \par}
    \vspace{1cm}
    {\large 26 November 2024 \par}
\end{titlepage}

\section{Introduction}

Unsteady open channel flows represent non-uniform, time-dependent flow patterns commonly found in natural and 
engineered water systems. Analyzing these flows is critical for designing efficient hydraulic systems and 
managing water resources. \\

\begin{figure}[H]
    \centering
    \includegraphics[width= 0.8\textwidth]{images/open.png}
    \caption{Open channel flow}
    \label{fig:sample}
\end{figure}

Open channel flow are conduits  in which the liquid flows with a free surface exposed to atmospheric pressure. 
Examples include rivers, canals, and drainage systems. \\

Understanding the behavior 
of flows in these channels is critical for applications in civil and environmental engineering, such as flood 
forecasting, sediment transport, and irrigation system design.\\

Numerical modeling of unsteady flow provides engineers with tools to simulate complex systems efficiently. Analytical 
solutions to the governing equations are often impossible due to non-linearities and boundary conditions. 
Numerical methods overcome these challenges, allowing accurate simulation of real-world scenario.\\

\noindent
In this paper we focus on:\\

\noindent
\textbf{1. Deriving and understanding the governing equations for unsteady open channel flow.}\\

\noindent
\textbf{2. Applying numerical techniques for discretizing and solving these equations.}\\

\noindent
\textbf{3. Implementing a Python-based solution to simulate the flow.}\\

\noindent
\textbf{4. Analyzing results and discussing}\\

\section{Governing Equations}
The unsteady state one-dimensional quasi-horizontal flows can be described in the terms of average 
velocity u(x,t) , or the flow discharge Q(x,t) and the water depth y(x,t). The two governing equations of the 
phenomenon are based on the principles of -\\


\noindent
\textbf{Mass conservation -}

\begin{figure}[H]
    \centering
    \includegraphics[width= 0.8\textwidth]{images/mass.png}
    \caption{Open channel flow}
    \label{fig:sample}
\end{figure}


\[ \frac{\partial A }{\partial t} + \frac{\partial Q}{\partial x} = q_L\]

For rectangular channel the equation can be written as - 

\[ \frac{\partial y}{\partial t} + \frac{1}{B} \frac{\partial Q}{\partial x} =  \frac{q_L}{B} \]
where \(q_L\) is the lateral inflow 

This equation expresses the evolution of free surface elevation in terms of the difference between the incoming 
and outgoing discharges through the infinitesimal control volume.\\

\noindent
\textbf{Momentum balance equation - }\\

\begin{figure}[H]
    \centering
    \includegraphics[width= 0.8\textwidth]{images/momentum.png}
    \caption{Open channel flow}
    \label{fig:sample}
\end{figure}

The following are taken into consideration -\\

The hydrstatic pressure forces acting on the two sides.
The component of the weight of water \(W_x\) within the control volume along the flow direction that is
parallel to bed.
The sum of the distributed frictional forces \(\tau_o\) over the wetted perimeter of the channel.\\

The equation takes the form -

\[ \frac{\partial u}{\partial t} + u \frac{\partial u}{\partial t} = -g \frac{\partial y}{\partial x} + g(S_o - S_e) \] 

where \(S_o\) is the slope of the channel bed and \( S_e = \frac{\tau_o}{g\rho R_h}\) is the slope of the
energy grade line. 
By using the chezy formula for wall friction - 

\[ S_e = \frac{u^2}{(C_z)^2 R_h} \] 

For rectangular channels , A = By and if the channel is much larger than depth ( as in river cross section geometry).
\(R_h = y\)\\

For \(q_L =0 \) and neglecting the non linear terms the governing equations are written as - 

\begin{tcolorbox}[colback=blue!20!white, colframe=black, width=\textwidth, boxrule=0.5mm, sharp corners, left=1mm, right=1mm, top=1mm, bottom=1mm]
\[ \frac{\partial y}{\partial t} + \frac{1}{B} \frac{\partial Q}{\partial x} = 0 \] 

\[ \frac{\partial u }{\partial t}  = -g \frac{\partial y}{\partial x}  + g (S_o - S_e) \] 
\end{tcolorbox}

\noindent
These are the two linear hyperbolic equations of the first order.\\

The discharge relation for rectangular channel of variable width - 

\[ Q = uBy = uA \] 

\section{Auxillary conditions}

\textbf{The initial conditions} - the values of u(x, t =0) and y(x,t = 0) should be provided foir the entire
solution domain. In case of an initially 'dry bed', the initial conditions are u(x, t =0) = 0 and y(x,t = 0) = 0.\\

\noindent
For the upstream end the incoming fllod discharge hydrograph Q(x = 0, t) is provided. It represents Q values
with an observation time interval \( \delta t\).\\

\noindent
\textbf{For the downstream end boundary - }\\

\noindent
1. A constant water depth can be maintained - y(x=L, t) = Y = constant\\
2. The flow can be led to a critical flow regime like free outflow and has the form of a critical 
relation between flow depth and outflow.

\section{Numerical solution approach}

The numerical solution of the system of the governing equations is done by the means of a staggered grid
We write an explicit finite difference scheme\\

\noindent
Forward difference for time derivatives - \\
\noindent
Central difference for space derivatives -

\begin{tcolorbox}[colback=pink!80!white, colframe=black, width=\textwidth, boxrule=0.5mm, sharp corners, left=1mm, right=1mm, top=1mm, bottom=1mm]
\begin{align*}
    \frac{y_i^{n+1} - y_i^n}{\Delta t} &= -\frac{2}{B_i + B_{i+1}} \frac{Q_{i+1}^n - Q_i^n}{\Delta x}, \\
    \frac{u_i^{n+1} - u_i^n}{\Delta t} &= -g \frac{y_i^{n+1} - y_{i-1}^{n+1}}{\Delta x} 
    + g \left( S_o - \frac{2 |u_i^n| u_i^n}{C_z^2 \left( y_i^{n+1} + y_{i-1}^{n+1} \right)} \right), \\
    Q_i^{n+1} &= u_i^{n+1} B_i \frac{y_i^{n+1} + y_{i-1}^{n+1}}{2}.
\end{align*}
\end{tcolorbox}

To maintain numerical stability the \textbf{CFL criterion} must be satisfied at all times.\\

\noindent
Free flow relationship for downstream is given by -  \( Q_out = CBy^(1.5) \)

\section{Solution to a problem }

Input data required to solve the problem of unsteady flow in rectangular open channel of constant width\\

\noindent
h = Flow depth m\\
b = Channel width (m)\\
S = Bed slope\\
C = Chezy coefficient of friction \\
Qin = Inflow hydrograph (\( m^3/s\))\\
Dx = Spatial step (m)\\
Dt = Time step (s)\\
nx = Number of spatial steps\\
nt = Number of time steps\\
dt = Time step of the inflow hydrograph (s)\\
nth = Number of time steps of the inflow hydrograph\\

\noindent
\begin{tcolorbox}[colback=yellow!30!white, colframe=black, width=\textwidth, boxrule=0.5mm, sharp corners, left=1mm, right=1mm, top=1mm, bottom=1mm]
\textbf{Problem 1}
\noindent 
- The problem is of routing of a flood wave over a constant width rectangular channel with an initial dry bed, \(y(x, t = 0) = 0\).
The data used for the simulation are the following:\\
\end{tcolorbox}

\noindent
 Channel width = 100 m\\
 Channel length = 41,000 m\\
 Bed slope = 0.001\\
 Chezys coefficient of friction = 50\\

The data for the inflow hydrograph upstream boundary condition is given.

\begin{longtable}{|c|c|}
    \hline
    \textbf{Time (seconds)} & \textbf{Discharge (m\(^3\)/s)} \\
    \hline
    \endfirsthead
    
    \hline
    0 & 0 \\
    2000 & 0 \\
    4000 & 600 \\
    6000 & 400 \\
    8000 & 200 \\
    10000 & 100 \\
    12000 & 50 \\
    14000 & 0 \\
    \hline
    
\end{longtable}

The downstream boundary condition was defined 
by a flow over weir relationship. The spatial and temporal discretization 
Steps were selected as \(\Delta x = 1000 m\) and \(\Delta t = 5 s\). Since the inflow hydrograph discretization step \(\partial t = 2000 s\) is different from that of the com
putational time step \(\Delta t = 5 s\) an interpolation is required.\\

\noindent
\textbf{Python code:}

\noindent
\textbf{Step1: Interpolation of inflow discharge}\\

\noindent
\textbf{Step2: Update velocity and discharge}\\

\noindent
\textbf{Step3: Downstream Boundary condition}\\

\noindent
\textbf{Step4: Update water depth}\\

\noindent
\textbf{Step5: Store data for plotting}\\

\noindent
\textbf{Step6: Plotting the results}\\

\begin{ColoredVerbatim}[formatcom=\color{red}]

    import numpy as np
    import matplotlib.pyplot as plt
    
    # Input data
    g = 9.81
    Dx = 1000
    Dt = 5
    nx = 41
    nt = 9000
    dt = 2000
    nth = 15
    
    # Initialize arrays
    h = np.zeros(nx)
    S = np.full(nx, 0.001)
    b = np.full(nx, 100)
    C = np.full(nx, 50)
    Qin = [0, 600, 400, 200, 100, 50, 0, 0, 0, 0, 0, 0, 0, 0, 0]
    u = np.zeros(nx)
    Q = np.zeros(nx)
    
    # Initialize variables for storing data at specific times
    h1t = h2t = h3t = h4t = h5t = h6t = np.zeros(nx)
    Q1t = Q2t = Q3t = Q4t = Q5t = Q6t = np.zeros(nx)
    
    # Main program
    for k in range(1, nt + 1):
        # Calculation of inflow discharge by interpolation
        jj = k * Dt / dt
        jr = int(np.round(jj)) + 1
        if jr >= nth:
            jr = nth - 1
        if jr < len(Qin) - 1:  # Only interpolate if jr+1 is within bounds
            Q[0] = Qin[jr] + (Qin[jr + 1] - Qin[jr]) * (k * Dt - (jr - 1) * dt) / dt
        else:
            Q[0] = Qin[jr]  # No interpolation if we are at the last index
        # Calculation of the velocity and discharge
        for i in range(1, nx - 1):
            if h[i - 1] < 0.001:
                u[i] = 0
            else:
                u[i] += Dt * g * (S[i] - (h[i] - h[i - 1]) / Dx - u[i] * abs(u[i]) / C[i]**2 * 2 / (h[i] + h[i - 1]))
                Q[i] = u[i] * b[i] * (h[i] + h[i - 1]) / 2
    
        # Downstream boundary condition
        Q[nx - 1] = b[nx - 1] * h[nx - 2] * np.sqrt(g * h[nx - 2])
    
        # Estimation of the water depths
        for i in range(nx - 1):
            h[i] -= Dt / Dx / (b[i] + b[i + 1]) * 2 * (Q[i + 1] - Q[i])
    
        # Store data at specific time intervals for plotting
        if k == round(nt / 9):
            h1t, Q1t = h.copy(), Q.copy()
        if k == round(nt / 6):
            h2t, Q2t = h.copy(), Q.copy()
        if k == round(nt / 4.5):
            h3t, Q3t = h.copy(), Q.copy()
        if k == round(nt / 3):
            h4t, Q4t = h.copy(), Q.copy()
        if k == round(nt / 2):
            h5t, Q5t = h.copy(), Q.copy()
        if k == nt:
            h6t, Q6t = h.copy(), Q.copy()
    
    # Plotting inflow hydrograph
    plt.figure()
    plt.plot(range(1, nth + 1), Qin, 'k', linewidth=1.5)
    plt.xlabel('Number of time steps × 2000 [s]')
    plt.ylabel('Inflow hydrograph [m³/s]')
    plt.axis([0, 15, 0, 660])
    plt.show()
    
    # Plotting water discharge
    plt.figure()
    plt.plot(range(1, nx + 1), Q1t, 'b', linewidth=1.5, label='t=1000×Dt')
    plt.plot(range(1, nx + 1), Q2t, 'r', linewidth=1.5, label='t=1500×Dt')
    plt.plot(range(1, nx + 1), Q3t, 'g', linewidth=1.5, label='t=2000×Dt')
    plt.plot(range(1, nx + 1), Q4t, 'b--', linewidth=1.5, label='t=3000×Dt')
    plt.plot(range(1, nx + 1), Q5t, 'r--', linewidth=1.5, label='t=4500×Dt')
    plt.plot(range(1, nx + 1), Q6t, 'g--', linewidth=1.5, label='t=9000×Dt')
    plt.xlabel('Longitudinal distance × 1000 [m]')
    plt.ylabel('Water discharge [m³/s]')
    plt.axis([0, 42, 0, 550])
    plt.legend()
    plt.show()
    
    # Plotting water depth
    plt.figure()
    plt.plot(range(1, nx + 1), h1t, 'b', linewidth=1.5, label='t=1000×Dt')
    plt.plot(range(1, nx + 1), h2t, 'r', linewidth=1.5, label='t=1500×Dt')
    plt.plot(range(1, nx + 1), h3t, 'g', linewidth=1.5, label='t=2000×Dt')
    plt.plot(range(1, nx + 1), h4t, 'b--', linewidth=1.5, label='t=3000×Dt')
    plt.plot(range(1, nx + 1), h5t, 'r--', linewidth=1.5, label='t=4500×Dt')
    plt.plot(range(1, nx + 1), h6t, 'g--', linewidth=1.5, label='t=9000×Dt')
    plt.xlabel('Longitudinal distance × 1000 [m]')
    plt.ylabel('Water depth [m]')
    plt.axis([0, 42, 0, 2.6])
    plt.legend()
    plt.show()
    
\end{ColoredVerbatim}

\begin{figure}[H]
    \centering
    \includegraphics[width= 0.8\textwidth]{images/11.png}
    \caption{Inflow Hydrograph}
    \label{fig:sample}
\end{figure}    
    
\begin{figure}[H]
    \centering
    \includegraphics[width= 0.8\textwidth]{images/12.png}
    \caption{Flow rates over the channel at different times}
    \label{fig:sample}
\end{figure}

\begin{figure}[H]
    \centering
    \includegraphics[width= 0.8\textwidth]{images/13.png}
    \caption{Water depths over the channel at different times}
    \label{fig:sample}
\end{figure}

\begin{tcolorbox}[colback=yellow!40!white, colframe=black, width=\textwidth, boxrule=0.5mm, sharp corners, left=1mm, right=1mm, top=1mm, bottom=1mm]
\textbf{Problem 2} 
-  Change the roughness coefficient linearly from Cz = 105 \(\sqrt{m}/s\) at the 
upstream boundary to Cz = 30 \(\sqrt{m}/s\) at the downstream boundary. 
\end{tcolorbox}


\noindent
In this script we need to incorporate a changing roughness coefficient, transitioning linearly from \(C_z = 105\) upstream to 30 downstream over the entire solution.\\

\noindent
\textbf{Pythoncode}

\begin{ColoredVerbatim}[formatcom=\color{red}]
    import numpy as np
    import matplotlib.pyplot as plt
    
    # Input data
    g = 9.81
    Dx = 1000
    Dt = 5
    nx = 41
    nt = 9000
    dt = 2000
    nth = 15
    
    # Initialize arrays
    h = np.zeros(nx)
    S = np.full(nx, 0.001)
    b = np.full(nx, 100)
    C = np.full(nx, 105)  # Initial value of C is 105
    Qin = [0, 600, 400, 200, 100, 50, 0, 0, 0, 0, 0, 0, 0, 0, 0]
    u = np.zeros(nx)
    Q = np.zeros(nx)
    
    # Initialize variables for storing data at specific times
    h1t = h2t = h3t = h4t = h5t = h6t = np.zeros(nx)
    Q1t = Q2t = Q3t = Q4t = Q5t = Q6t = np.zeros(nx)
    
    # Main program
    for k in range(1, nt + 1):
        # Update C linearly from 105 to 30
        C[:] = 105 - (105 - 30) * k / nt  # Update C for all positions
        
        # Calculation of inflow discharge by interpolation
        jj = k * Dt / dt
        jr = int(np.round(jj)) + 1
        if jr >= nth:
            jr = nth - 1
        if jr < len(Qin) - 1:  # Only interpolate if jr+1 is within bounds
            Q[0] = Qin[jr] + (Qin[jr + 1] - Qin[jr]) * (k * Dt - (jr - 1) * dt) / dt
        else:
            Q[0] = Qin[jr]  # No interpolation if we are at the last index
        
        # Calculation of the velocity and discharge
        for i in range(1, nx - 1):
            if h[i - 1] < 0.001:
                u[i] = 0
            else:
                u[i] += Dt * g * (S[i] - (h[i] - h[i - 1]) / Dx - u[i] * abs(u[i]) / C[i]**2 * 2 / (h[i] + h[i - 1]))
                Q[i] = u[i] * b[i] * (h[i] + h[i - 1]) / 2
    
        # Downstream boundary condition
        Q[nx - 1] = b[nx - 1] * h[nx - 2] * np.sqrt(g * h[nx - 2])
    
        # Estimation of the water depths
        for i in range(nx - 1):
            h[i] -= Dt / Dx / (b[i] + b[i + 1]) * 2 * (Q[i + 1] - Q[i])
    
        # Store data at specific time intervals for plotting
        if k == round(nt / 9):
            h1t, Q1t = h.copy(), Q.copy()
        if k == round(nt / 6):
            h2t, Q2t = h.copy(), Q.copy()
        if k == round(nt / 4.5):
            h3t, Q3t = h.copy(), Q.copy()
        if k == round(nt / 3):
            h4t, Q4t = h.copy(), Q.copy()
        if k == round(nt / 2):
            h5t, Q5t = h.copy(), Q.copy()
        if k == nt:
            h6t, Q6t = h.copy(), Q.copy()
    
    # Plotting inflow hydrograph
    plt.figure()
    plt.plot(range(1, nth + 1), Qin, 'k', linewidth=1.5)
    plt.xlabel('Number of time steps × 2000 [s]')
    plt.ylabel('Inflow hydrograph [m³/s]')
    plt.axis([0, 15, 0, 660])
    plt.show()
    
    # Plotting water discharge
    plt.figure()
    plt.plot(range(1, nx + 1), Q1t, 'b', linewidth=1.5, label='t=1000×Dt')
    plt.plot(range(1, nx + 1), Q2t, 'r', linewidth=1.5, label='t=1500×Dt')
    plt.plot(range(1, nx + 1), Q3t, 'g', linewidth=1.5, label='t=2000×Dt')
    plt.plot(range(1, nx + 1), Q4t, 'b--', linewidth=1.5, label='t=3000×Dt')
    plt.plot(range(1, nx + 1), Q5t, 'r--', linewidth=1.5, label='t=4500×Dt')
    plt.plot(range(1, nx + 1), Q6t, 'g--', linewidth=1.5, label='t=9000×Dt')
    plt.xlabel('Longitudinal distance × 1000 [m]')
    plt.ylabel('Water discharge [m³/s]')
    plt.axis([0, 42, 0, 550])
    plt.legend()
    plt.show()
    
    # Plotting water depth
    plt.figure()
    plt.plot(range(1, nx + 1), h1t, 'b', linewidth=1.5, label='t=1000×Dt')
    plt.plot(range(1, nx + 1), h2t, 'r', linewidth=1.5, label='t=1500×Dt')
    plt.plot(range(1, nx + 1), h3t, 'g', linewidth=1.5, label='t=2000×Dt')
    plt.plot(range(1, nx + 1), h4t, 'b--', linewidth=1.5, label='t=3000×Dt')
    plt.plot(range(1, nx + 1), h5t, 'r--', linewidth=1.5, label='t=4500×Dt')
    plt.plot(range(1, nx + 1), h6t, 'g--', linewidth=1.5, label='t=9000×Dt')
    plt.xlabel('Longitudinal distance × 1000 [m]')
    plt.ylabel('Water depth [m]')
    plt.axis([0, 42, 0, 2.6])
    plt.legend()
    plt.show()

\end{ColoredVerbatim}

\begin{figure}[H]
    \centering
    \includegraphics[width= 0.8\textwidth]{images/21.png}
    \caption{Inflow Hydrograph}
    \label{fig:sample}
\end{figure}    
    
\begin{figure}[H]
    \centering
    \includegraphics[width= 0.8\textwidth]{images/22.png}
    \caption{Flow rates over the channel at different times}
    \label{fig:sample}
\end{figure}

\begin{figure}[H]
    \centering
    \includegraphics[width= 0.8\textwidth]{images/23.png}
    \caption{Water depths over the channel at different times}
    \label{fig:sample}
\end{figure}

\section{Results and conclusion}

\begin{itemize}
    \item The study successfully demonstrates the application of numerical methods to solve the Saint-Venant equations for unsteady open channel flows, providing insights into flood routing and channel hydraulics.
    \item The explicit finite difference scheme with staggered grids proved efficient and accurate for simulating transient flows under varying conditions.
    \item Boundary conditions, such as the upstream inflow hydrograph and downstream weir relationships, play critical roles in shaping flow dynamics.
    \item Changes in roughness coefficients significantly impact the energy dissipation and flow profiles, reinforcing the need to account for spatially varying friction in hydraulic modeling.
    \item The presented Python implementation serves as a versatile and reproducible tool for solving real-world hydraulic problems, offering potential for further refinement and application in complex systems.
\end{itemize}

\section{Analytical vs numerical solution}

When solving unsteady open channel flow problems numerically, the solution deviates from the actual physical flow due to inherent limitations and approximations in the numerical approach.
This is due to the following reasons - \\

\noindent
\textbf{Discretization Error:}\\
Spatial (grid size) and temporal (time step) resolution affect accuracy.
Larger grid sizes or time steps result in smoothed approximations of wavefronts.\\

\noindent
\textbf{Numerical Diffusion:}\\
Artificial smoothing of steep gradients (e.g., wavefronts), causing wave energy to spread out unrealistically.
Leads to a more gradual wave propagation than in reality.\\

\noindent
\textbf{Truncation Error:}\\
Results from approximating derivatives in the governing equations using finite differences.
Higher-order schemes reduce this error but require more computational effort.\\

\noindent
\textbf{Boundary Condition Approximations:}\\
Simplified representations of upstream inflow and downstream outflow may not fully capture dynamic boundary effects.\\

\noindent
\textbf{Friction and Roughness Modeling:}\\
Numerical friction models (e.g., Chezy or Manning's coefficient) assume constant values, which may not reflect real, variable roughness.\\

\noindent
The differences we will see in the numerical unsteady open channel flow solution are - \\

1. The analytical solution will have sharper wavefronts which will be well defined and steep. While in the numerical solution wavefronts
will be smoothned due to numerical diffusion.\\

2. Energy loss representation - In analytical solution it varies with time and location while we have modeled them as uniform friction losses.\\

3. Flow transitions - The flow transitions will be sudden(hydraulic jumps) while our numerical solution will show them gradual due to smoothing.\\

4. Turbulenfce effects are neglected or over simplified.

\end{document}