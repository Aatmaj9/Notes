\documentclass{book}
\usepackage{amsmath} % For advanced mathematical typesetting
\usepackage{graphicx} % For including graphics
\usepackage{amssymb} % For mathematical symbols
\usepackage[table]{xcolor}
\usepackage{tcolorbox}
\usepackage{float}
\usepackage{array}

\begin{document} 
\tableofcontents
\part{Mechanical Vibrations}
\chapter{Vibration of sinlge DOF System}
\section{Free Undamped harmonic motion}
Let us consider freee undamped vibration of SDOF>

\[ m \ddot{x} + kx =0\]

Let \( \omega^2 = k/m\)
\[ \ddot{x}(t) + \omega^2 x(t) = 0\]

Let us assume the solution to be - \( x(t) = Ce^{\lambda t}\)
Substituting into equation of motion we get - 
\[ (\lambda^2 + \omega^2) C e^{\lambda t} = 0\] 
The equation is valid for all t if - 
\[ \lambda^2 + \omega^2 = 0\]
\[ \lambda = \pm i\omega \]
\[ x(t) = C_1e^{i\omega t} + C_2 e^{-i \omega t}\] 
\[ x(t) = (C_1 + C_2) cos \omega t + i (C_1 - C_2) sin \omega t\]

\(C_1\) and \( C_2\) are constants and both are complex numbers but since x(t)should be real \(C_1\) and \(C_2\) should be complex conjugates.

\[ C_1 = a+ib \ \  and \ \ C_2 = a-ib\]
\[ x(t) = A cos \omega t + B sin \omega t\]

We can get these constants from the inital conitions. x(t) = 0 and \(\dot{x}(0) = 0\)

\[ A = x(0) \ \ and \ \ B = \dot{x}(0)/ \omega \]

We can rewrite the solution in the form - 

\[ x(t) = C cos( \omega t - \phi) \] 

where 

\[ C = \sqrt{A^2 + B^2 } \ \ and \ \ \phi = tan^{-1}\frac{B}{A} \]


\section{Driven undamped harmonic motion}
\[m\ddot{x} + kx = F(t)\]
\[\ddot{x} + \omega^2{x} = F(t)/m\]
The general solution of this equation is superposition of two solutions - Particular solutioon and complementary solution.\\

\noindent
Particular solution \( x_p(t)\) is obtained by solving the equation \(\ddot{x} + \omega^2{x} = F(t)/m \)\\


\noindent
Complimentary solution \(x_c(t)\) is obtained by solving the equation \(\ddot{x} + \omega^2{x} = 0 \)\\

\noindent
Consider the external force \( F(t) = F_d \ sin \ \omega_d t \)\\
We assume particular solution of the form - 

\[ x_p(t) = A \ sin \ \omega_d t \]
\[ \ddot{x}_p(t) =- A \omega_d^2 \ sin \ \omega_d t \]

\[ - A \omega_d^2 \ sin \ \omega_d t + \omega^2  A \ sin \ \omega_d t = F sin (\omega_d t) \] 
where \(F = F_d/m\)

\[ A = \frac{F}{ \omega^2 - \omega_d^2} = \frac{F_d}{m\omega^2(1 - (\frac{\omega_d}{\omega})^2 )} = \frac{F_d}{k ( 1 - \beta^2)} \]

\[ x_p(t) = \frac{F_d}{k ( 1 - \beta^2)} sin \ (\omega_d t )\]

The displaceement corresponding to the static application of the external load \( F_d\) is called statuc displacement.
\[ x_{st} = \frac{F_d}{k}\]
Dynamic amplication factor or magnification factor is ratio of amplitude of motion to the static displacement.

Dynamic response factor - 
\begin{tcolorbox}[colback=yellow!40!white, colframe=black, width=\textwidth, boxrule=0.5mm, sharp corners, left=1mm, right=1mm, top=1mm, bottom=1mm]
\[ R_d = \frac{ 1}{1 - \beta^2} \]
\end{tcolorbox}

\section{Free Damped harmonic motion}
Let a mass m be subjected to drag force proportional to velocity \(F_f = -bv\)
\[ F = -b\dot{x} -kx\]
\[ m\ddot{x} + b\dot{x} +kx = 0\]
\[ \ddot{x} + 2\gamma \dot{x} + \omega^2x =0\]
We guess the solution to be \(x=Ae^{\alpha t}\)
\[ \alpha^2 + 2\gamma \alpha + \omega^2 =0\]
\[ \alpha = - \gamma \pm \sqrt{\gamma^2 - \omega^2}\]
Sum of 2 different solutions is again a solution to a equation is the property of linear differential equation.
Hence the general solution is -
\[ x(t) = Ae^{\alpha_1 t} + Be^{\alpha_2 t}\]
\[ x(t) = e^{-\gamma t}(Ae^{t\sqrt{\gamma^2 - \omega^2}} + Be^{-t\sqrt{\gamma^2 - \omega^2}})\]
\[ x(t) = e^{-\gamma t}(Ae^{\Omega t} + Be^{-\Omega t}) \ \  where \ \ \Omega^2 = \gamma^2 - \omega^2\]

\subsection{Underdamping}
Underdamping occurs when \(\Omega^2 < 0\). Since \(\Omega\) is imaginary we define \(\Omega = i \overline{\omega} = i\sqrt{\omega^2 - i^2}\)
\[ x(t) =e^{-\gamma t} (Ae^{i\overline{\omega} t} + Be^{-i\overline{\omega} t})\]
\[x(t) = C e^{-\gamma t} cos(\overline{\omega} t + \phi_0)\]
\subsection{Overdamping}

Overdamping occurs when  \(\Omega^2 > 0\).  \(\Omega\) is real and positive.

\[ x(t)  = e^{-\gamma t}(Ae^{\Omega t} + Be^{-\Omega t}) \]
There is no oscillatory motion in this case.
\subsection{Critical damping}
In this case \(\Omega =0\) and \(\gamma = \omega\). We have only one solution here \( x(t) = e^{- \gamma t} \)\\

\noindent
Note: From theory of differential equations we have another solution - \( x(t) = t e^{ -\gamma t} \) 

\[ x(t) = e^{-\gamma t}( A + Bt) \]

For critical damping the motion converges to 0 the quickest, for overdamping the motion is slower. In both these cases no oscillation is present.
Only in underdamping oscillatory motion is present.



\section{Driven damped harmonic motion}
The system is driven by a external force F(t).

The equation which represents driven damped harmonic motion is -
\[ m\ddot{x} + b\dot{x} +kx = F(t)\]
We can rewrite the above equation as - 
\[ \ddot{x} + 2\gamma \dot{x} + \omega^2 x = f(t)\]
where f(t) = F(t)/m.


Let us consider \(f(t) = C_0e^{i\omega_0 t}\) for the ease of solving
We get the general solution for x by the priciple of superposition - by adding the solutions for the equations
\[ \ddot{x} + 2\gamma \dot{x} + \omega^2 x = C_0e^{i\omega_0 t} \ \ and \ \ \ddot{x} + 2\gamma \dot{x} + \omega^2 x = 0  \]

We try thr solution \(x(t) = Ae^{i\omega_0 t}\) for solving the equation - 
\[\ddot{x} + 2\gamma \dot{x} + \omega^2 x = C_0e^{i\omega_0 t}\]
Substituting and solving we get - 
\[ A = \left(\frac{C_0}{-\omega_0^2 + 2i\gamma \omega + \omega^2}\right) \]
\[ x_p(t) = \left(\frac{C_0}{-\omega_0^2 + 2i\gamma \omega + \omega^2}\right) e^{iw_0t}\]
This called the particular solution and is denoted by \(x_p(t)\) while the other solution is called
complimentary solution and is denoted by \(x_c(t)\). We have already computed general solution for \(x_c(t)\) 

\[x(t) = x_c(t)  + x_p(t)\]
Hnece we can write the general solution as - 

\begin{tcolorbox}[colback=pink!30!white, colframe=black, width=\textwidth, boxrule=0.5mm, sharp corners, left=1mm, right=1mm, top=1mm, bottom=1mm]
    \[ x(t) =  e^{-\gamma t}(Ae^{\Omega t} + Be^{-\Omega t}) + \left(\frac{C_0}{-\omega_0^2 + 2i\gamma \omega + \omega^2}\right) e^{iw_0t}\]
\end{tcolorbox}

We are more interested in a more general form of driving force \(F(t) = F_d cos(w_d t)\). Let us see how to go about solving such harmonic motion.

\[ \ddot{x} + 2\gamma \dot{x} + \omega^2 x = \frac{F(t)}{m}\]
\[ \ddot{x} + 2\gamma \dot{x} + \omega^2 x = \frac{F_d}{m} cos(w_d t)\]
\[ \ddot{x} + 2\gamma \dot{x} + \omega^2 x = F cos(w_d t)\]
\[ \ddot{x} + 2\gamma \dot{x} + \omega^2 x = \frac{F}{2} e^{iw_d t} + \frac{F}{2} e^{-iw_d t} \]

Using method of superposition we can add the solutions for the two terms on the RHS 
\[ x_p(t)= \left( \frac{F/2}{-\omega_d^2 + 2 i \gamma \omega_d + \omega^2} \right) e^{i\omega_d t} + \left( \frac{F/2}{-\omega_d^2 - 2 i \gamma \omega_d + \omega^2} \right) e^{-i\omega_d t}\]
\[ x_p(t) = \left( \frac{F(\omega^2 - \omega_d^2)}{(\omega^2 - \omega_d^2) + 4\gamma^2 \omega_d^2} + \right) cos(\omega_d t) + \left( \frac{2F\gamma \omega_d}{(\omega^2 - \omega_d^2) + 4\gamma^2 \omega_d^2} \right) sin(w_d t)\]
Note that x(t) is real as it describes position.
We now define :
\[ R = \sqrt{(\omega^2 - \omega_d^2)^2 + (2 \gamma \omega_d)^2}\]
\[x_p(t) = \frac{F}{R^2} \left[  (\omega^2 - \omega_d^2) cos(\omega_d t) + 2 \gamma \omega_d sin(w_d t)\right] \]
\[ x_p(t) = \frac{F}{R} \left[  \frac{(\omega^2 - \omega_d^2)}{R} cos(\omega_d t) + \frac{2 \gamma \omega_d}{R} sin(w_d t)\right] \]
\begin{tcolorbox}[colback=pink!30!white, colframe=black, width=\textwidth, boxrule=0.5mm, sharp corners, left=1mm, right=1mm, top=1mm, bottom=1mm]
    \[ x_p(t) = \frac{F}{R} cos(\omega_d t - \phi)\] 
\end{tcolorbox}
where we have - 
\[ cos \phi = \frac{\omega^2 - \omega_d^2}{R} \ , \ sin \phi = \frac{2 \gamma \omega_d}{R} \ , \ tan \phi = \frac{2 \gamma \omega_d}{\omega^2 - \omega_d^2}\]
\begin{tcolorbox}[colback=pink!30!white, colframe=black, width=\textwidth, boxrule=0.5mm, sharp corners, left=1mm, right=1mm, top=1mm, bottom=1mm]
\[ x(t)= C e^{-\gamma t} cos(\overline{\omega} t + \phi_0) + \frac{F}{R} cos(\omega_d t - \phi)\]
For the case of underdamping.
\end{tcolorbox}

Note that the constant C here is calculated by using initial conditions.\\

Let us write the solution x(t) in the form - 
\begin{tcolorbox}[colback=pink!30!white, colframe=black, width=\textwidth, boxrule=0.5mm, sharp corners, left=1mm, right=1mm, top=1mm, bottom=1mm]
\[ x(t) =  e^{- \gamma t}[Acos(\overline{\omega} t) + B sin(\overline{\omega } t)] + \frac{F}{R} cos(\omega_d t - \phi)\]
\end{tcolorbox}

Now we calculate constsnts A dn B from initial conditions : \(x(0) = \dot{x}(0) = 0\)
\[ x(0) = A + \frac{F}{R} cos \phi \] 
\begin{tcolorbox}[colback=blue!20!white, colframe=black, width=\textwidth, boxrule=0.5mm, sharp corners, left=1mm, right=1mm, top=1mm, bottom=1mm]
\[ A = x(0) - \frac{F}{R}  cos \phi\] 
\end{tcolorbox}
\[ \dot{x}(t) =  -\gamma e^{- \gamma t}[ \ Acos\ \overline{\omega} t + B sin \ \overline{\omega } t \ ] + e^{\gamma t} [ \  - A \ \overline{\omega} \ sin \ \overline{\omega} t + \ B \ \overline{\omega} \ cos \ \overline{\omega} t] - \frac{F\omega_d}{R} sin(\omega_d t - \phi)\]
\[ \dot{x}(0) = -\gamma A + B \ \overline{\omega} + \frac{F \omega_d}{R} sin \phi\]
\[ \dot{x}(0) = -\gamma \left[ x(0) - \frac{F}{R} cos \phi \right] + B \ \overline{\omega} + \frac{F \omega_d}{R} sin \phi\]
\begin{tcolorbox}[colback=blue!20!white, colframe=black, width=\textwidth, boxrule=0.5mm, sharp corners, left=1mm, right=1mm, top=1mm, bottom=1mm]
\[ B= \frac{\dot{x}(0) + \gamma x(0)} {\overline{\omega}} - \frac{F}{R \ \overline{\omega}} (\gamma cos \phi + \omega_d sin \phi) \]
\end{tcolorbox}

\noindent
\textbf{Let us now look at the driving force of the form \(F(t) = F_d sin(\omega_d t) \)}

\[ \ddot{x} + 2\gamma \dot{x} + \omega^2 x = \frac{F(t)}{m}\]
\[ \ddot{x} + 2\gamma \dot{x} + \omega^2 x = \frac{F_d}{m} sin(w_d t)\]
\[ \ddot{x} + 2\gamma \dot{x} + \omega^2 x = F sin(w_d t)\]
\[ \ddot{x} + 2\gamma \dot{x} + \omega^2 x = \frac{F}{2i} e^{iw_d t} - \frac{F}{2i} e^{-iw_d t} \]

Using method of superposition we can add the solutions for the two terms on the RHS 
\[ x_p(t)= \left( \frac{F/2i}{-\omega_d^2 + 2 i \gamma \omega_d + \omega^2} \right) e^{i\omega_d t} - \left( \frac{F/2i}{-\omega_d^2 - 2 i \gamma \omega_d + \omega^2} \right) e^{-i\omega_d t}\]
\[ x_p(t) = \left( \frac{F(\omega^2 - \omega_d^2)}{(\omega^2 - \omega_d^2) + 4\gamma^2 \omega_d^2} \right) sin(\omega_d t) - \left( \frac{2F\gamma \omega_d}{(\omega^2 - \omega_d^2) + 4\gamma^2 \omega_d^2} \right) cos(w_d t)\]
Note that x(t) is real as it describes position.
We now define :
\[ R = \sqrt{(\omega^2 - \omega_d^2)^2 + (2 \gamma \omega_d)^2}\]
\[x_p(t) = \frac{F}{R^2} \left[  (\omega^2 - \omega_d^2) sin(\omega_d t) - 2 \gamma \omega_d cos(w_d t)\right] \]
\[ x_p(t) = \frac{F}{R} \left[  \frac{(\omega^2 - \omega_d^2)}{R} sin(\omega_d t) - \frac{2 \gamma \omega_d}{R} cos(w_d t)\right] \]
\begin{tcolorbox}[colback=pink!30!white, colframe=black, width=\textwidth, boxrule=0.5mm, sharp corners, left=1mm, right=1mm, top=1mm, bottom=1mm]
    \[ x_p(t) = \frac{F}{R} sin(\omega_d t - \phi)\] 
\end{tcolorbox}
where we have - 
\[ cos \phi = \frac{\omega^2 - \omega_d^2}{R} \ , \ sin \phi = \frac{2 \gamma \omega_d}{R} \ , \ tan \phi = \frac{2 \gamma \omega_d}{\omega^2 - \omega_d^2}\]

\noindent
\textbf{The above particular solution can also be derived from an alternate method - }

Assume the solution in the form-
\[ x_p(t) = C cos \omega_d t + D sin \omega_d t \] 
\[ \dot{x}_p(t)= -C\omega_d \ sin \omega_d t + D \omega_d \ cos \omega_d t \]
\[ \ddot{x}_p(t) = - C \omega_d^2 \ cos \omega_d t - D \omega_d^2 \ sin \omega_d t\]
Substitute these in our equation - 
\[ \ddot{x} + 2\gamma \dot{x} + \omega^2 x = F sin(w_d t)\]
\noindent
Equating cos and sin terms on lhs and rhs we get :
\[ D( \omega^2 - \omega_d^2) - 2\gamma C \omega_d = F\] 
\[ C (\omega^2 - \omega_d^2) +2 \gamma D \omega_d =0\] 
Solving these 2 equations for C and D we have - 
\[ C = - \frac{2 F \gamma \omega_d }{R^2}\] 
\[ D = \frac{F}{R^2} ( \omega^2 - \omega_d^2) \]

Hence we have - 
\[ x_p(t) =  \frac{F}{R^2} ( \omega^2 - \omega_d^2)\ sin (\omega_d t)  - \frac{2 F \gamma \omega_d }{R^2} cos (\omega_d t)\]
Which matches with the solution we derived earlier !

\begin{tcolorbox}[colback=pink!30!white, colframe=black, width=\textwidth, boxrule=0.5mm, sharp corners, left=1mm, right=1mm, top=1mm, bottom=1mm]
\[ x(t)= C e^{-\gamma t} cos(\overline{\omega} t + \phi_0) + \frac{F}{R} sin(\omega_d t - \phi)\]
For the case of underdamping.
\end{tcolorbox}

\noindent
If there is damping in the system i.e \(\gamma > 0\) then the homogenous part of the solution goes 
to zero for large t ans we are left only with particular solution. Hnece the system approaches 
definite x(t) namely \(x_p(t) \) independent of initial conditions.

\noindent
Let us write the solution x(t) in the form - 
\begin{tcolorbox}[colback=pink!30!white, colframe=black, width=\textwidth, boxrule=0.5mm, sharp corners, left=1mm, right=1mm, top=1mm, bottom=1mm]
\[ x(t) =  e^{- \gamma t}[Acos(\overline{\omega} t) + B sin(\overline{\omega } t)] + \frac{F}{R} sin(\omega_d t - \phi)\]
\end{tcolorbox}

Now we calculate constsnts A dn B from initial conditions : \(x(0) = \dot{x}(0) = 0\)
\[ x(0) = A - \frac{F}{R} sin \phi \] 
\begin{tcolorbox}[colback=blue!20!white, colframe=black, width=\textwidth, boxrule=0.5mm, sharp corners, left=1mm, right=1mm, top=1mm, bottom=1mm]
\[ A = x(0) + \frac{F}{R}  sin \phi\] 
\end{tcolorbox}
\[ \dot{x}(t) =  -\gamma e^{- \gamma t}[ \ Acos\ \overline{\omega} t + B sin \ \overline{\omega } t \ ] + e^{\gamma t} [ \  - A \ \overline{\omega} \ sin \ \overline{\omega} t + \ B \ \overline{\omega} \ cos \ \overline{\omega} t] + \frac{F\omega_d}{R} cos(\omega_d t - \phi)\]
\[ \dot{x}(0) = -\gamma A + B \ \overline{\omega} + \frac{F \omega_d}{R} cos \phi\]
\[ \dot{x}(0) = -\gamma \left[ x(0) + \frac{F}{R} sin \phi \right] + B \ \overline{\omega} + \frac{F \omega_d}{R} cos \phi\]
\begin{tcolorbox}[colback=blue!20!white, colframe=black, width=\textwidth, boxrule=0.5mm, sharp corners, left=1mm, right=1mm, top=1mm, bottom=1mm]
\[ B= \frac{\dot{x}(0) + \gamma x(0)} {\overline{\omega}} + \frac{F}{R \ \overline{\omega}} (\gamma sin \phi - \omega_d cos \phi) \]
\end{tcolorbox}


The displaceement corresponding to the static application of the external load \( F_d\) is called statuc displacement.
\[ x_{st} = \frac{F_d}{k}\]
Dynamic amplication factor or magnification factor is ratio of amplitude of motion to the static displacement.

Dynamic response factor - 
\begin{tcolorbox}[colback=yellow!40!white, colframe=black, width=\textwidth, boxrule=0.5mm, sharp corners, left=1mm, right=1mm, top=1mm, bottom=1mm]
\[ R_d = \frac{ F/R}{F_d/k} = \frac{\omega^2}{R} \]
\end{tcolorbox}

\noindent
Let us define frequency ratio - ratio between forcing freuency and natural frequency - 
\[ \beta = \frac{\omega_d}{\omega} \]

\section{Physical interpretation }
Let us look at the steady state solution.
The amplitude of steady state solution is proportional to 1/R.\\


\noindent
Given \( \omega_d\) and \(\gamma\) the amplitude is maximum when \( \omega = \omega_d\).\\

\noindent
Given \( \omega\) and \(\gamma\) the amplitude is maximum when \( \omega_d = \sqrt{\omega^2  - 2 \gamma^2 }\).
In the case of weak damping ( \(\gamma << 0\)) the maximum is achieved when \(\omega = \omega_d\). \\

Let us see how \(R_d\) and \(\phi\) differ as a function of \(\beta\) and different values of \(\gamma\)\\

\noindent
\( 1.\ \ \beta \rightarrow 0 \ \ \ \omega_d \rightarrow 0 \ \ \ R \rightarrow \omega^2 \ \ \ R_d \rightarrow 1 \) irrespective of \(\gamma \)\\

\noindent
\( 2. \ \ \beta \rightarrow \infty \ \ \  \omega \rightarrow 0 \ \ \ R_d \rightarrow 0\) irrespective of \(\gamma \)\\

\noindent
\( 3. \ \ \beta \rightarrow 1 \ \ \  \omega \rightarrow \omega_d \ \  R \rightarrow 2 \gamma \omega_d \) and \(R_d\) is finite for \(\gamma > 0\)\\

\noindent
\(4. \ \  \beta \rightarrow 1 \ \ \ R \rightarrow 0 \ \ \ R_d \rightarrow \infty \) for \(\gamma \rightarrow 0\) \\

\noindent
Now we seperate into 3 regions when excitation frequency is less than natural 
frequency and excitation frequency is equal to the natural frequency and excitation frequency is 
greater than natural frequency, the following physical interpretation can be made with the help of 
the force equilibrium with respect to phase angle.\\

\( 1.\ \ \beta << 1 \ \ \ \omega >> \omega_d \ \ , \phi \) is small \\ - 

The external load and motion is in phase. The response in this frequency range is controlled 
by the stiffness of the system. The effect of inertia force and damping force will be 
small.\\

\( 2.\ \ \beta = 1 \ \ \ \omega = \omega_d \ \ , \phi = \pi/2 \) \\ - 

The external load is in phase with velocity and so the response will be dominated by the damping force. For higher damping value the amplitude will be small.

\( 3.\ \ \beta >> 1 \ \ \ \omega << \omega_d \ \ , \phi \) is close to \(\pi\) \\ - 

The external load and motion will be out of phase.  The load is in phase with accelaration and so the response will be controlled by the mass of the system.

\chapter{Two Degree of Freedom System}
\chapter{Multidegree of Freedom System}
\chapter{Determination of natural frequencies and mode shapes}
\chapter{Continuos systems}
\chapter{Numerical Integration methods in vibration analysis}

\part{Motion in waves}
\chapter{Ship response to regular waves}
\end{document}