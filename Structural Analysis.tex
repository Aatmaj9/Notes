\documentclass{report}
\usepackage{amsmath} % For advanced mathematical typesetting
\usepackage{graphicx} % For including graphics
\usepackage{amssymb} % For mathematical symbols
\usepackage[table]{xcolor}
\usepackage{tcolorbox}
\usepackage{float}
\usepackage{array}
\begin{document}
\tableofcontents
\chapter{Stress Strain}
\section{Stress}
\section{Strain}
\chapter{Stress strain transformation}
\chapter{Axial load}
\chapter{Torsion}
\chapter{Bending}
\chapter{Deflection of beams and shafts}
The deflection of beam or shaft must often be limited in order to provide integrity and
stability of a structure
\section{Moment-curvature Relationship}
We will now develop an important relationship between the internal moment and the radius of curvature \(\rho\) of the 
elastic curve. Due to loading the deformation of the beam is caused by both internal shear force and bending moment. If the 
beam has length that is much greater than its depth the greatest deformation will be caused by bending and therefore we will
direct our attention towards its effects. We will look at deflections caused by shear later.


When the internal moment M deforms the element of the beam, the angle between the cross sections becomes \(d\theta\) 
The arc dx represents portion of the elastic curve that intersects the neutral axis. Any arc on the element other than dx is subjected to normal strain.
Let us see strain in ds located at position y from neutral axis.

\[ dx=ds= \rho d\theta \]
\[ ds = (\rho - y) d\theta\]
\[ \epsilon = \frac{(\rho - y)d\theta - \rho d\theta}{\rho d\theta} \]
\[ \epsilon = \frac{-y}{\rho} \]
From flexture formula - 
\[ \sigma =  \frac{-My}{I} \]
From Hookes law - 
\[ \epsilon = \frac{\sigma}{E}\] 

Hence - 
\begin{tcolorbox}[colback=pink!30!white, colframe=black, width=\textwidth, boxrule=0.5mm, sharp corners, left=1mm, right=1mm, top=1mm, bottom=1mm]
    \[\frac{1}{\rho} = \frac{M}{EI}\] 
\end{tcolorbox}

The equation of elastic curve of beam can be represented mathematically as v = f(x). To obtain this equation we must first represent the
curvature \((1/\rho )\) in terms of v and x. From calculus we have the relationship - 

\[ \frac{1}{\rho} = \frac{ d^2v/dx^2}{[1+ (dv/dx)^2]^{3/2}} \] 
The slope of the elastic curve determined from dv/dx will be very small. Hence we approximate - 

\[ \frac{d^2v}{dx^2} = \frac{M}{EI} \]

It is also possible to write this equation in alternative forms. We substitue V = dM/dx to get - 

\[ EI \frac{d^3v}{dx^3} = V(x)\] 
Differentiating again using w = dV/dx - 
\[ EI \frac{d^4v}{dx^4} = w(x) \]

Solution of any of these equations requires sucessive integration to
obtain the deflection v of the elastic curve. For each integration it is 
necessary to introduce a constant of integration and then solve for all 
constants to obtain a unique solution for a particular problem.

\subsection{Sign convention and coordinates}


\section{Boundary and continuity conditions}

The constants 
\section{Simple Integration method}

\textbf{Simply supported beam with load P acting at center -}\\

\noindent
For \(0 < x <l/2\)

\[V(x) = P/2\]
\[M(x) = Px/2\]
\[ EI \frac{d^2v}{dx^2} = \frac{Px}{2} \] 
\[ EI \frac{dv}{dx} = \frac{Px^2}{4} + c_1 \] 
\[ EI v_1(x) = \frac{Px^3}{12} + c_1 x + c_2 \] 

Applying boundary condition \(v_1(x = 0) = 0 \). We get \(c_2 =0\).

\[ v_1(x) = \frac{1}{EI} \left( \frac{Px^3}{12} + c_1 x \right) \]

For \(0 < x <l/2\)

\[V(x) = -P/2 \] 
\[ M(x) = Pl/2 + V(x).x \] 
\[ M(x) = \frac{P}{2}(l-x) \]
\[ EI \frac{d^2v}{dx^2} = \frac{P}{2}(l-x)\]
\[ EI \frac{dv}{dx} = \frac{P}{2}\left(lx-\frac{x^2}{2}\right) + c_3\]
\[ EI v_2(x) = \frac{P}{2}\left(\frac{lx^2}{2}-\frac{x^3}{6}\right) + c_3 x + c_4\]

Applying the other three boundary conditions to get \(c_1\) \(c_3\) and \( c_4\)\\

1. \(v_1(x) = v_2(x)\) at \( x =l/2 \)
\[ \left( \frac{Pl^3}{96} + \frac{c_1l}{2}  \right) = \frac{P}{2}\left( \frac{l^3}{8} - \frac{l^3}{48}\right) + \frac{c_3l}{2} + c_4 \]
\[ \frac{Pl^3}{24} = \frac{c_1l}{2} - \frac{c_3l}{2} - c_4 \] 


2. \(v_2(x) =0\) at \(x=l \)
\[ \frac{P}{2}\left( \frac{l^3}{2} - \frac{l^3}{6}\right) + c_3 l + c_4= 0\]
\[ \frac{Pl^3}{6} + c_3 l + c_4= 0\]
\[ c_4 = -c_3l - \frac{Pl^3}{6} \]

3. \(\theta_1(x) = \theta_2(x)\) at \(x=l/2 \)
\[ \frac{Pl^2}{16} + c_1 = \frac{P}{2}\left( \frac{l^2}{2}- \frac{l^2}{8} \right) + c_3 \]
\[ c_1 = \frac{Pl^2}{8} + c_3 \]

From 3 and 2 substitute \(c_1\) and \(c_4\) in 1 to get \(c_3\)

\[ \frac{Pl^3}{24} = \frac{Pl^3}{16} + \frac{c_3l}{2} - \frac{c_3l}{2}+ c_3l + \frac{Pl^3}{6} \]

\begin{tcolorbox}[colback=pink!30!white, colframe=black, width=\textwidth, boxrule=0.5mm, sharp corners, left=1mm, right=1mm, top=1mm, bottom=1mm]
  \[ c_3 = -\frac{3Pl^2}{16} \ \ and \ \ c_1 = -\frac{Pl^2}{16} \ \ and \ \  c_4 = \frac{Pl^3}{48}\]
\end{tcolorbox}

The deflection of beam ar x= l/2 comes - 
\begin{tcolorbox}[colback=blue!20!white, colframe=black, width=\textwidth, boxrule=0.5mm, sharp corners, left=1mm, right=1mm, top=1mm, bottom=1mm]
  \[ v = -\frac{Pl^3}{48 EI} \]
\end{tcolorbox}

\noindent
\textbf{Cantilevered beam with load P acting at distance x -}\\
\section{Macaulays method}

\section{Moment area method}
\section{Castiglianos Theorem}
\section{Method of superposition}
\section{Statistically indeterminate beams}
\section{Matrix method of analysis}
For analysis by the matrix stiffness method the continuous beam is modeled as a series
of straight prismatic members connected  at their ends to joints, so that external reactions
act only at the joints. It is important to realize that because the joints are modeled
as rigid joints they satisfy the continuity and restraint conditions of the actual structure

Degrees of Freedom
The degrees of freedom of a beam are simply its unknown joint displacements (translations and rotations)
. Since axial deformations in a beam are neglected, a joint can have two degrees of freedom
namely translation in y direction and rotation about z-axis. Thus the number of structure
coordiantes at a joint of a beam equals 2.

We calculate the total degree of freedom of a beam by summing the DOF of all n joints
in the beam. The number of degrees of freedom of a beam are represented by NDOF,
The NDOF \(\times\) joint displacement vector d can be written as

\[
  d_{NDOF\times1} =
  \left[ {\begin{array}{c}
    d_1 \\
    d_2 \\
    \vdots \\
    d_n \\
  \end{array} } \right]
\]

We can also obtain NDOF by 
\[NDOF = 2(NJ) - NR\]
where NJ is number of joints of the beam and NR denotes the number of joint 
displacements restrained by supports (or number of restrained co-ordinates).

While analysing beam it is not necessary to draw its deformed shap. Instead we draw
a line diagram specifying all the structure coordinates (DOF and restrained coordinates)
by assigning numbers to the arrows drawn at the joints in the directions of joint displacements
The DOF are numbered first starting at the lowest numbered joint and proceeding sequentially
to the highest numbered joint. If a joint has two DOF then the translation in Y direction
is numbered first followed by rotation. After all DOF of beam are numbered we number the 
restrained coordinates beginning with a number equal to NDOF + 1. Starting at the lowest 
numbered joint and proceeding sequentially to the highest number of joint, all the 
restrained coordinates of the beam are numbered. The y coordinate corresponding
to the reaction force is numbered first followedby the rotation coordinate 
corresponding to reaction couple . The number assigned to the last restrained coordinate
of the beam is always 2(NJ).

The external loads on beams may be applied at joints as well as on the members. The 
external loads applied at joints at the joints are referred to joint loads whereas
external loads acting between the ends of the members are termed member loads. In this
section we focus our attention only on joint loads.

A beam with n DOF can be subjected to maximum n joint loads.We can write the joint 
load vector P as 

\[
  d_{NDOF\times1} =
  \left[ {\begin{array}{c}
    P_1 \\
    P_2 \\
    \vdots \\
    P_n \\
  \end{array} } \right]
\]




\chapter{Bucking of columns}
\section{Columns Having Various Types of Supports}
Effective length - 
\end{document}